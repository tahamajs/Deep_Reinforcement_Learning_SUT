\documentclass[12pt]{article}
\usepackage[a4paper, top=2.5cm, bottom=2.5cm, left=1.5cm, right=1.5cm]{geometry}
\usepackage{amsmath, amsfonts, amssymb, mathtools}
\usepackage{fancyhdr, setspace, parskip}
\usepackage{graphicx, caption, subfig, array, multirow}
\usepackage{hyperref, enumitem, cancel}
\usepackage[T1]{fontenc}
\usepackage{tgtermes}
\usepackage[dvipsnames]{xcolor}
\usepackage{tocloft}
\usepackage{titlesec}
\usepackage{lipsum}  

\definecolor{DarkBlue}{RGB}{10, 0, 80}

% Hyperlink setup
\hypersetup{
    colorlinks=true,
    linkcolor=DarkBlue,
    filecolor=BrickRed,      
    urlcolor=RoyalBlue,
}


% Header and footer customization
\fancyhead{}
\fancyhead[L]{
{\fontfamily{lmss}{\color{DarkBlue}
\textbf{\leftmark}
}}
}
\fancyhead[R]{
{\fontfamily{ppl}\selectfont {\color{DarkBlue}
{Deep RL Course [Spring 2025]}
}}
}

\fancyfoot{}
\fancyfoot[C]{
{\fontfamily{lmss}{\color{BrickRed}
\textbf{\thepage}
}}
}

\renewcommand{\sectionmark}[1]{ \markboth{\thesection\quad #1}{} }

\renewcommand{\headrule}{{\color{BrickRed}\hrule width\headwidth height 0.5pt}}
\renewcommand{\footrulewidth}{0pt}


% Table of Contents customizations
\renewcommand{\cftsecafterpnum}{\vskip6pt}
\renewcommand{\cftsubsecafterpnum}{\vskip3pt}
\renewcommand{\cftsubsubsecafterpnum}{\vskip3pt}
\renewcommand{\cftsecfont}{\sffamily\large}
\renewcommand{\cftsubsecfont}{\sffamily}
\renewcommand{\cftsubsubsecfont}{\sffamily}
% \renewcommand{\cftsecdotsep}{1}
\renewcommand{\cftsubsecdotsep}{1}
\renewcommand{\cftsubsubsecdotsep}{1}


% Section title styles
\titleformat*{\section}{\LARGE\bfseries\color{DarkBlue}}
\titleformat*{\subsection}{\Large\bfseries\color{DarkBlue}}
\titleformat*{\subsubsection}{\large\bfseries\color{DarkBlue}}

\definecolor{light-gray}{gray}{0.95}
\newcommand{\code}[1]{\colorbox{light-gray}{\texttt{#1}}}

% Start of the document
\pagestyle{fancy}

%%%%%%%%%%%%%%%%%%%%%%%%%%%%%%%%%%%%%%%%%%%%%%%%%

\begin{document}

\pagenumbering{gobble}
\thispagestyle{plain}

\begin{center}

\vspace*{-1.5cm}
\begin{figure}[!h]
    \centering
    \includegraphics[width=0.7\linewidth]{figs/cover-std.png}
\end{figure}

{
\fontfamily{ppl}

{\color{DarkBlue} {\fontsize{30}{50} \textbf{
Deep Reinforcement Learning
}}}

{\color{DarkBlue} {\Large
Professor Mohammad Hossein Rohban
}}
}


\vspace{20pt}

{
\fontfamily{lmss}


{\color{RedOrange}
{\Large
Solution for Homework [9]
}\\
}
{\color{BrickRed}
\rule{12cm}{0.5pt}

{\Huge
[Advanced RL Algorithms]
}
\rule{12cm}{0.5pt}
}

\vspace{10pt}

{\color{RoyalPurple} { \small By:} } \\
\vspace{10pt}

{\color{Blue} { \LARGE [Full Name] } } \\
\vspace{5pt}
{\color{RoyalBlue} { \Large [Student Number] } }


\vspace*{\fill}
\begin{center}
\begin{tabular}{ccc}
    \includegraphics[width=0.14\linewidth]{figs/sharif-logo.png} & \includegraphics[width=0.14\linewidth]{figs/riml-logo.png} & \includegraphics[width=0.14\linewidth]{figs/dlr-logo.png} \\
\end{tabular}
\end{center}


\vspace*{-.25cm}

{\color{YellowOrange} {
\rule{10cm}{0.5pt} \\
\vspace{2pt}
\large Spring 2025}
}}
\vspace*{-1cm}

\end{center}

%%%%%%%%%%%%%%%%%%%%%%%%%%%%%%%%%%%%%%%%%%%%%%%%%

\newpage
\pagenumbering{gobble}
\thispagestyle{plain}
{\fontfamily{lmss}\selectfont {\color{BrickRed} \textbf{\tableofcontents} }}


%%%%%%%%%%%%%%%%%%%%%%%%%%%%%%%%%%%%%%%%%%%%%%%%%

\newpage
\pagenumbering{arabic}

{\fontfamily{lmss}\selectfont {\color{DarkBlue}
\noindent\rule{\textwidth}{1.5pt}

\section{Distributional Reinforcement Learning[40-points]}
\noindent\rule{\textwidth}{1.5pt}
\noindent\rule{\textwidth}{0.2pt}

\subsection{Theoretical Foundation[15-points]}
\subsubsection{a)[8-points]} Explain the fundamental difference between traditional value-based RL and distributional RL. Why is modeling the full return distribution beneficial?

\textbf{Answer:}

Traditional value-based reinforcement learning methods, such as DQN and Q-learning, focus on estimating the expected value of returns:

\begin{equation}
Q(s,a) = \mathbb{E}[R_t | s_t = s, a_t = a]
\end{equation}

In contrast, distributional RL models the entire probability distribution of returns rather than just the expectation:

\begin{equation}
Z(s,a) \text{ represents the full distribution of returns}
\end{equation}
\begin{equation}
Q(s,a) = \mathbb{E}[Z(s,a)]
\end{equation}

\textbf{Key Benefits:}
\begin{enumerate}
\item \textbf{Richer Representation}: Captures uncertainty and risk in returns
\item \textbf{Multi-Modal Returns}: Can represent multiple outcome scenarios
\item \textbf{Improved Learning}: Provides more informative learning signal
\item \textbf{Better Stability}: Reduces variance in value estimation
\item \textbf{Risk-Sensitive Policies}: Enables risk-aware decision making
\end{enumerate}

\subsubsection{b)[7-points]} Consider two actions with the same expected value but different distributions:
\begin{itemize}
\item Action A: Always returns 10 (deterministic)
\item Action B: Returns 0 or 20 with equal probability
\end{itemize}
Both have E[R] = 10, but how does distributional RL distinguish their risk profiles?

\textbf{Answer:}

Both actions have the same expected value $\mathbb{E}[R] = 10$, but distributional RL can distinguish their risk profiles:

\textbf{Action A (Deterministic):}
\begin{itemize}
\item Distribution: $\delta_{10}$ (point mass at 10)
\item Variance: $\text{Var}[R] = 0$
\item Risk: No uncertainty, guaranteed outcome
\end{itemize}

\textbf{Action B (Stochastic):}
\begin{itemize}
\item Distribution: $0.5 \cdot \delta_0 + 0.5 \cdot \delta_{20}$
\item Variance: $\text{Var}[R] = 100$
\item Risk: High uncertainty, potential for both loss and gain
\end{itemize}

\textbf{How Distributional RL Distinguishes:}
\begin{enumerate}
\item \textbf{Risk Assessment}: Action B has higher variance, indicating higher risk
\item \textbf{Tail Behavior}: Action B can produce extreme outcomes (0 or 20)
\item \textbf{Policy Selection}: Risk-averse agents might prefer Action A, risk-seeking agents might prefer Action B
\item \textbf{Conditional Value at Risk (CVaR)}: Can compute risk measures like CVaR$_{0.1}$ to assess worst-case scenarios
\end{enumerate}

This distinction is impossible with traditional value-based methods that only consider expected values.

\noindent\rule{\textwidth}{0.2pt}

\subsection{C51 Algorithm[15-points]}
\subsubsection{a)[8-points]} Describe the C51 algorithm in detail. How does it represent and update return distributions? Include the projection step.

\subsubsection{b)[7-points]} Implement the projection algorithm for C51. Show how to project the Bellman-updated distribution back onto the fixed support.

\noindent\rule{\textwidth}{0.2pt}

\subsection{Quantile Regression DQN[10-points]}
\subsubsection{a)[5-points]} Explain QR-DQN and how it differs from C51. What are the advantages of using quantile regression?

\subsubsection{b)[5-points]} Implement the quantile Huber loss function for QR-DQN.

\noindent\rule{\textwidth}{0.2pt}
}}

%%%%%%%%%%%%%%%%%%%%%%%%%%%%%%%%%%%%%%%%%%%%%%%%%

\newpage

{\fontfamily{lmss}\selectfont {\color{DarkBlue}
\noindent\rule{\textwidth}{1.5pt}
\section{Rainbow DQN[50-points]}
\noindent\rule{\textwidth}{1.5pt}
\noindent\rule{\textwidth}{0.2pt}

\subsection{Rainbow Components[30-points]}
\subsubsection{a)[5-points]} List and briefly describe the six components that Rainbow DQN combines.

\subsubsection{b)[8-points]} Explain how Double Q-Learning reduces overestimation bias in DQN.

\subsubsection{c)[8-points]} Describe Prioritized Experience Replay. How does it improve sample efficiency?

\subsubsection{d)[9-points]} Implement the Dueling Network architecture. Explain why mean subtraction is used in the combination.

\noindent\rule{\textwidth}{0.2pt}

\subsection{Integration and Implementation[20-points]}
\subsubsection{a)[10-points]} Show how to integrate all six Rainbow components in a single architecture.

\subsubsection{b)[10-points]} What are the main implementation challenges in Rainbow DQN? How can they be addressed?

\noindent\rule{\textwidth}{0.2pt}
}}

%%%%%%%%%%%%%%%%%%%%%%%%%%%%%%%%%%%%%%%%%%%%%%%%%

\newpage

{\fontfamily{lmss}\selectfont {\color{DarkBlue}
\noindent\rule{\textwidth}{1.5pt}
\section{Twin Delayed DDPG (TD3)[40-points]}
\noindent\rule{\textwidth}{1.5pt}
\noindent\rule{\textwidth}{0.2pt}

\subsection{Core Innovations[20-points]}
\subsubsection{a)[7-points]} Explain the three key innovations in TD3 and why each is necessary.

\subsubsection{b)[6-points]} Why does taking the minimum of two Q-networks reduce overestimation?

\subsubsection{c)[7-points]} Describe target policy smoothing and its theoretical justification.

\noindent\rule{\textwidth}{0.2pt}

\subsection{Algorithm Implementation[20-points]}
\subsubsection{a)[10-points]} Provide complete pseudocode for TD3 and explain the key differences from DDPG.

\subsubsection{b)[10-points]} Analyze the contribution of each TD3 component through ablation studies.

\noindent\rule{\textwidth}{0.2pt}
}}

%%%%%%%%%%%%%%%%%%%%%%%%%%%%%%%%%%%%%%%%%%%%%%%%%

\newpage

{\fontfamily{lmss}\selectfont {\color{DarkBlue}
\noindent\rule{\textwidth}{1.5pt}
\section{Trust Region Policy Optimization (TRPO)[35-points]}
\noindent\rule{\textwidth}{1.5pt}
\noindent\rule{\textwidth}{0.2pt}

\subsection{Trust Region Concept[15-points]}
\subsubsection{a)[8-points]} Explain the trust region concept in policy optimization. Why is it important?

\subsubsection{b)[7-points]} What is the natural policy gradient? How does it relate to TRPO?

\noindent\rule{\textwidth}{0.2pt}

\subsection{TRPO Algorithm[20-points]}
\subsubsection{a)[10-points]} Provide complete TRPO algorithm with all implementation details.

\subsubsection{b)[10-points]} Implement the conjugate gradient method for computing natural gradients.

\noindent\rule{\textwidth}{0.2pt}
}}

%%%%%%%%%%%%%%%%%%%%%%%%%%%%%%%%%%%%%%%%%%%%%%%%%

\newpage

{\fontfamily{lmss}\selectfont {\color{DarkBlue}
\noindent\rule{\textwidth}{1.5pt}
\section{Advanced Value Functions[25-points]}
\noindent\rule{\textwidth}{1.5pt}
\noindent\rule{\textwidth}{0.2pt}

\subsection{Dueling Networks[15-points]}
\subsubsection{a)[8-points]} Explain the dueling network architecture. Why is it beneficial?

\subsubsection{b)[7-points]} Implement the dueling DQN and explain the mean subtraction technique.

\noindent\rule{\textwidth}{0.2pt}

\subsection{Retrace(λ)[10-points]}
\subsubsection{a)[5-points]} Explain the Retrace(λ) algorithm and its advantages for off-policy learning.

\subsubsection{b)[5-points]} Implement the Retrace(λ) target computation.

\noindent\rule{\textwidth}{0.2pt}



%%%%%%%%%%%%%%%%%%%%%%%%%%%%%%%%%%%%%%%%%%%%%%%%%



%%%%%%%%%%%%%%%%%%%%%%%%%%%%%%%%%%%%%%%%%%%%%%%%%

\end{document}