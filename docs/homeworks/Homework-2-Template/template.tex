\documentclass[12pt]{article}
\usepackage[a4paper, top=2.5cm, bottom=2.5cm, left=1.5cm, right=1.5cm]{geometry}
\usepackage{amsmath, amsfonts, amssymb, mathtools}
\usepackage{fancyhdr, setspace, parskip}
\usepackage{graphicx, caption, subfig, array, multirow}
\usepackage{hyperref, enumitem, cancel}
\usepackage[T1]{fontenc}
\usepackage{tgtermes}
\usepackage[dvipsnames]{xcolor}
\usepackage{tocloft}
\usepackage{titlesec}
\usepackage{lipsum}  

\definecolor{DarkBlue}{RGB}{10, 0, 80}

% Hyperlink setup
\hypersetup{
    colorlinks=true,
    linkcolor=DarkBlue,
    filecolor=BrickRed,      
    urlcolor=RoyalBlue,
}


% Header and footer customization
\fancyhead{}
\fancyhead[L]{
{\fontfamily{lmss}{\color{DarkBlue}
\textbf{\leftmark}
}}
}
\fancyhead[R]{
{\fontfamily{ppl}\selectfont {\color{DarkBlue}
{Deep Reinforcement Learning [Spring 2025]}
}}
}

\fancyfoot{}
\fancyfoot[C]{
{\fontfamily{lmss}{\color{BrickRed}
\textbf{\thepage}
}}
}

\renewcommand{\sectionmark}[1]{ \markboth{\thesection\quad #1}{} }

\renewcommand{\headrule}{{\color{BrickRed}\hrule width\headwidth height 0.5pt}}
\renewcommand{\footrulewidth}{0pt}


% Table of Contents customizations
\renewcommand{\cftsecafterpnum}{\vskip6pt}
\renewcommand{\cftsubsecafterpnum}{\vskip3pt}
\renewcommand{\cftsubsubsecafterpnum}{\vskip3pt}
\renewcommand{\cftsecfont}{\sffamily\large}
\renewcommand{\cftsubsecfont}{\sffamily}
\renewcommand{\cftsubsubsecfont}{\sffamily}
% \renewcommand{\cftsecdotsep}{1}
\renewcommand{\cftsubsecdotsep}{1}
\renewcommand{\cftsubsubsecdotsep}{1}


% Section title styles
\titleformat*{\section}{\LARGE\bfseries\color{DarkBlue}}
\titleformat*{\subsection}{\Large\bfseries\color{DarkBlue}}
\titleformat*{\subsubsection}{\large\bfseries\color{DarkBlue}}

\definecolor{light-gray}{gray}{0.95}
\newcommand{\code}[1]{\colorbox{light-gray}{\texttt{#1}}}

% Start of the document
\pagestyle{fancy}

%%%%%%%%%%%%%%%%%%%%%%%%%%%%%%%%%%%%%%%%%%%%%%%%%

\begin{document}

\pagenumbering{gobble}
\thispagestyle{plain}

\begin{center}

\vspace*{-1.5cm}
\begin{figure}[!h]
    \centering
    \includegraphics[width=0.7\linewidth]{figs/cover-std.png}
\end{figure}

{
\fontfamily{ppl}

{\color{DarkBlue} {\fontsize{30}{50} \textbf{
Deep Reinforcement Learning
}}}

{\color{DarkBlue} {\Large
Professor Mohammad Hossein Rohban
}}
}


\vspace{20pt}

{
\fontfamily{lmss}


{\color{RedOrange}
{\Large
Homework 2:
}\\
}
{\color{BrickRed}
\rule{12cm}{0.5pt}

{\Huge
Value-Based Methods
}
\rule{12cm}{0.5pt}
}

\vspace{10pt}

{\color{RoyalPurple} { \small By:} } \\
\vspace{10pt}

{\color{Blue} { \LARGE [Full Name] } } \\
\vspace{5pt}
{\color{RoyalBlue} { \Large [Student Number] } }


\vspace*{\fill}
\begin{center}
\begin{tabular}{ccc}
    \includegraphics[width=0.14\linewidth]{figs/sharif-logo.png} & \includegraphics[width=0.14\linewidth]{figs/riml-logo.png} & \includegraphics[width=0.14\linewidth]{figs/dlr-logo.png} \\
\end{tabular}
\end{center}

\vspace*{-.25cm}

{\color{YellowOrange} {
\rule{10cm}{0.5pt} \\
\vspace{2pt}
\large Spring 2025}
}}w
\vspace*{-1cm}

\end{center}

%%%%%%%%%%%%%%%%%%%%%%%%%%%%%%%%%%%%%%%%%%%%%%%%%


\newpage
\pagenumbering{gobble}
\thispagestyle{plain}
{\fontfamily{lmss}\selectfont {\color{BrickRed} \textbf{\tableofcontents} }}

{\fontfamily{lmss}\selectfont {\color{DarkBlue}

\newpage

\subsection*{Grading}

The grading will be based on the following criteria, with a total of 100 points:

\[
\begin{array}{|l|l|}
\hline
\textbf{Task} & \textbf{Points} \\
\hline
\text{Task 1: Epsilon Greedy \& N-step Sarsa/Q-learning} & 40 \\
\hline
\quad \text{Jupyter Notebook} & 25 \\
\quad \text{Analysis and Deduction} & 15 \\
\hline
\text{Task 2: DQN vs. DDQN} & 50 \\
\hline
\quad \text{Jupyter Notebook} & 30 \\
\quad \text{Analysis and Deduction} & 20 \\
\hline
\text{Clarity and Quality of Code} & 5 \\
\text{Clarity and Quality of Report} & 5 \\
\hline
\text{Bonus 1: Writing your report in Latex } & 10 \\
\hline
\end{array}
\]

\textbf{Notes:}
\begin{itemize}
    \item Include well-commented code and relevant plots in your notebook.
    \item Clearly present all comparisons and analyses in your report.
    \item Ensure reproducibility by specifying all dependencies and configurations.
\end{itemize}

}

%%%%%%%%%%%%%%%%%%%%%%%%%%%%%%%%%%%%%%%%%%%%%%%%%

\newpage
\pagenumbering{arabic}

{\fontfamily{lmss}\selectfont {\color{DarkBlue}

% \section{N-Step Sarsa and N-Step Q-learning}
\section{Epsilon Greedy}
\subsection{Epsilon 0.1 initially has a high regret rate but decreases quickly. Why is that? [2.5-points]}
\subsection{Both epsilon 0.1 and 0.5 show jumps. What is the reason for this? [2.5-points]}
\subsection{Epsilon 0.9 changes linearly. Why? [2.5-points]}

\subsection{Compare the policy for epsilon values 0.1 and 0.9. How do they differ, and why do they look different? [2.5-points]}

\subsection{In the epsilon decay section, analyze the optimal policy for the row adjacent to the cliff (the lowest row). Then, compare the different learned policies and their corresponding rewards. [2.5-points]}

\section{N-step Sarsa and N-step Q-learning}
\subsection{What is the difference between Q-learning and sarsa? [2.5-points]}
\subsection{Compare how different values of n affect each algorithm's performance separately. [2.5-points]}
\subsection{Is a Higher or Lower n Always Better? Explain the advantages and disadvantages of both low and high n values. [2.5-points]}

}}

%%%%%%%%%%%%%%%%%%%%%%%%%%%%%%%%%%%%%%%%%%%%%%%%%

\newpage

{\fontfamily{lmss}\selectfont {\color{DarkBlue}

\section{DQN vs. DDQN}

\subsection{Which algorithm performs better and why? [3-points]}

\subsection{Which algorithm has a tighter upper and lower bound for rewards. [2-points]}
\subsection{Based on your previous answer, can we conclude that this algorithm exhibits greater stability in
learning? Explain your reasoning.
 [2-points]}
\subsection{What are the general issues with DQN?
 [2-points]}
\subsection{How can some of these issues be mitigated? (You may refer to external sources such as research
papers and blog posts be sure to cite them properly.)
 [3-points]}
\subsection{Based on the plotted values in the notebook, can the main purpose of DDQN be observed in the
results?
 [2-points]}
\subsection{The DDQN paper states that different environments influence the algorithm in various ways. Explain
these characteristics (e.g., complexity, dynamics of the environment) and their impact on DDQN\textquotesingle s performance. Then, compare them to the CartPole environment. Does CartPole exhibit these
characteristics or not? [4-points]}
\subsection{How do you think DQN can be further improved? (This question is for your own analysis, but
you may refer to external sources such as research papers and blog posts be sure to cite them
properly.) [2-points]}



}}




%%%%%%%%%%%%%%%%%%%%%%%%%%%%%%%%%%%%%%%%%%%%%%%%%

\newpage

{\fontfamily{lmss}\selectfont {\color{DarkBlue}

\begin{thebibliography}{9}

\bibitem{SuttonBarto}
R. Sutton and A. Barto, Reinforcement Learning: An Introduction, 2nd Edition, 2020. Available: \href{http://incompleteideas.net/book/the-book-2nd.html}{http://incompleteideas.net/book/the-book-2nd.html}.

\bibitem{SuttonBarto}
Gymnasium Documentation. Available: \href{https://gymnasium.farama.org/}{https://gymnasium.farama.org/}

\bibitem{SuttonBarto}
Grokking Deep Reinforcement Learning. Available: \href{https://www.manning.com/books/grokking-deep-reinforcement-learning}{https://www.manning.com/books/grokking-deep-reinforcement-learning}

\bibitem{SuttonBarto}
Deep Reinforcement Learning with Double Q-learning. Available: \href{https://arxiv.org/abs/1509.06461}{https://arxiv.org/abs/1509.06461}

\bibitem{SuttonBarto}
\href{https://www.freepik.com/free-vector/cute-artificial-intelligence-robot-isometric-icon_16717130.htm}{Cover image designed by freepik}

\end{thebibliography}

%%%%%%%%%%%%%%%%%%%%%%%%%%%%%%%%%%%%%%%%%%%%%%%%%

\end{document}
