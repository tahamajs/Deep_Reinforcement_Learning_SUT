\documentclass[12pt]{article}
\usepackage[a4paper, top=2.5cm, bottom=2.5cm, left=1.5cm, right=1.5cm]{geometry}
\usepackage{amsmath, amsfonts, amssymb, mathtools}
\usepackage{fancyhdr, setspace, parskip}
\usepackage{graphicx, caption, subfig, array, multirow}
\usepackage{hyperref, enumitem, cancel}
\usepackage[T1]{fontenc}
\usepackage{tgtermes}
\usepackage[dvipsnames]{xcolor}
\usepackage{tocloft}
\usepackage{titlesec}
\usepackage{lipsum}  
\usepackage{booktabs}

\definecolor{DarkBlue}{RGB}{10, 0, 80}

% Hyperlink setup
\hypersetup{
    colorlinks=true,
    linkcolor=DarkBlue,
    filecolor=BrickRed,      
    urlcolor=RoyalBlue,
}


% Header and footer customization
\fancyhead{}
\fancyhead[L]{
{\fontfamily{lmss}{\color{DarkBlue}
\textbf{\leftmark}
}}
}
\fancyhead[R]{
{\fontfamily{ppl}\selectfont {\color{DarkBlue}
{Deep RL [Spring 2025]}
}}
}

\fancyfoot{}
\fancyfoot[C]{
{\fontfamily{lmss}{\color{BrickRed}
\textbf{\thepage}
}}
}

\renewcommand{\sectionmark}[1]{ \markboth{\thesection\quad #1}{} }

\renewcommand{\headrule}{{\color{BrickRed}\hrule width\headwidth height 0.5pt}}
\renewcommand{\footrulewidth}{0pt}


% Table of Contents customizations
\renewcommand{\cftsecafterpnum}{\vskip6pt}
\renewcommand{\cftsubsecafterpnum}{\vskip3pt}
\renewcommand{\cftsubsubsecafterpnum}{\vskip3pt}
\renewcommand{\cftsecfont}{\sffamily\large}
\renewcommand{\cftsubsecfont}{\sffamily}
\renewcommand{\cftsubsubsecfont}{\sffamily}
% \renewcommand{\cftsecdotsep}{1}
\renewcommand{\cftsubsecdotsep}{1}
\renewcommand{\cftsubsubsecdotsep}{1}


% Section title styles
\titleformat*{\section}{\LARGE\bfseries\color{DarkBlue}}
\titleformat*{\subsection}{\Large\bfseries\color{DarkBlue}}
\titleformat*{\subsubsection}{\large\bfseries\color{DarkBlue}}

\definecolor{light-gray}{gray}{0.95}
\newcommand{\code}[1]{\colorbox{light-gray}{\texttt{#1}}}

% Start of the document
\pagestyle{fancy}

%%%%%%%%%%%%%%%%%%%%%%%%%%%%%%%%%%%%%%%%%%%%%%%%%

\begin{document}

\pagenumbering{gobble}
\thispagestyle{plain}

\begin{center}

\vspace*{-1.5cm}
\begin{figure}[!h]
    \centering
    \includegraphics[width=0.7\linewidth]{figs/cover-std.png}
\end{figure}

{
\fontfamily{ppl}

{\color{DarkBlue} {\fontsize{30}{50} \textbf{
Deep Reinforcement Learning
}}}

{\color{DarkBlue} {\Large
Professor Mohammad Hossein Rohban
}}
}


\vspace{20pt}

{
\fontfamily{lmss}


{\color{RedOrange}
{\Large
Homework 10:
}\\
}
{\color{BrickRed}
\rule{12cm}{0.5pt}

{\Huge
Exploration in Deep Reinforcement Learning
}
\rule{12cm}{0.5pt}
}

\vspace{10pt}

{\color{RoyalPurple} { \small By:} } \\
\vspace{10pt}

{\color{Blue} { \LARGE [Full Name] } } \\
\vspace{5pt}
{\color{RoyalBlue} { \Large [Student Number] } }


\vspace*{\fill}
\begin{center}
\begin{tabular}{ccc}
    \includegraphics[width=0.14\linewidth]{figs/sharif-logo.png} & \includegraphics[width=0.14\linewidth]{figs/riml-logo.png} & \includegraphics[width=0.14\linewidth]{figs/dlr-logo.png} \\
\end{tabular}
\end{center}


\vspace*{-.25cm}

{\color{YellowOrange} {
\rule{10cm}{0.5pt} \\
\vspace{2pt}
\large Spring 2025}
}}
\vspace*{-1cm}

\end{center}

%%%%%%%%%%%%%%%%%%%%%%%%%%%%%%%%%%%%%%%%%%%%%%%%%


\newpage
\pagenumbering{gobble}
\thispagestyle{plain}
{\fontfamily{lmss}\selectfont {\color{BrickRed} \textbf{\tableofcontents} }}

{\fontfamily{lmss}\selectfont {\color{DarkBlue}

\subsection*{Grading}

The grading will be based on the following criteria, with a total of 290 points:

\[
\begin{array}{|l|l|}
\hline
\textbf{Task} & \textbf{Points} \\
\hline
\text{Task 1: Bootstrap DQN Variants} & 100 \\
\text{Task 2: Random Network Distillation (RND)} & 100 \
 \\
\hline
\text{Clarity and Quality of Code} & 5 \\
\text{Clarity and Quality of Report} & 5 \\
\hline
\text{Bonus 1 } & 80 \\
\hline
\end{array}
\]

}



%%%%%%%%%%%%%%%%%%%%%%%%%%%%%%%%%%%%%%%%%%%%%%%%%

\newpage
\pagenumbering{arabic}

{\fontfamily{lmss}\selectfont {\color{DarkBlue}

\section{Task 1: Bootstrap DQN Variants}
\begin{itemize}[noitemsep]
    \item The complete guidelines for implementing the Bootstrap DQN algorithm, including the RPF and BIV
variants, are provided in the Jupyter notebook. You will find detailed instructions on how to set up the
environment, implement the algorithms, and evaluate their performance. 
    \item Make sure to read Guidelines
section in the notebook carefully.
    \end{itemize}

\section{Task 2: Random Network Distillation (RND)}
\begin{itemize}[noitemsep]
    \item You will implement the missing core components of Random Network Distillation (RND) combined with
a Proximal Policy Optimization (PPO) agent inside the MiniGrid environment.
    \item \textbf{TODO:} You must complete the following parts:
    \begin{table}[h]
    \centering
    \renewcommand{\arraystretch}{1.3}
    \begin{tabular}{ll}
    \toprule
    \textbf{File} & \textbf{TODO Description} \\
    \midrule
    \texttt{Core/model.py} & Implement the architecture of \texttt{TargetModel} and \texttt{PredictorModel}. \\
    \texttt{Core/model.py} & Implement \texttt{\_init\_weights()} method for proper initialization. \\
    \texttt{Core/ppo\_rnd\_agent.py} & Implement \texttt{calculate\_int\_rewards()} to compute intrinsic rewards. \\
    \texttt{Core/ppo\_rnd\_agent.py} & Implement \texttt{calculate\_rnd\_loss()} to compute predictor training loss. \\
    \bottomrule
    \end{tabular}
    \caption{Summary of required TODO implementations}
    \end{table}
    
    \item Questions:
    \begin{enumerate}
    \item What is the intuition behind Random Network Distillation (RND)? Why does a prediction error signal encourage better exploration?
    
    \item Why is it beneficial to use both intrinsic and extrinsic returns in the PPO loss function?
    
    \item What happens when you increase the \texttt{predictor\_proportion} (i.e., the proportion of masked features used in the RND loss)? Does it help or hurt learning?
    
    \item Try training with \texttt{int\_adv\_coeff=0} (removing intrinsic motivation). How does the agent's behavior and reward change?
    
    \item Inspect the TensorBoard logs. During successful runs, how do intrinsic rewards evolve over time? Are they higher in early training?
\end{enumerate}
\end{itemize}


}}

\end{document}