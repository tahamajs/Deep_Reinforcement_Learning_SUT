%%%%%%%%%%%%%%%%%%%%%%%%%%%%%%%%%%%%%%%%%
% Wenneker Assignment
% LaTeX Template
% Version 2.0 (12/1/2019)
%
% This template originates from:
% http://www.LaTeXTemplates.com
%
% Authors:
% Vel (vel@LaTeXTemplates.com)
% Frits Wenneker
%
% License:
% CC BY-NC-SA 3.0 (http://creativecommons.org/licenses/by-nc-sa/3.0/)
% 
%%%%%%%%%%%%%%%%%%%%%%%%%%%%%%%%%%%%%%%%%

%----------------------------------------------------------------------------------------
%	PACKAGES AND OTHER DOCUMENT CONFIGURATIONS
%----------------------------------------------------------------------------------------

\documentclass[11pt]{scrartcl} % Font size

\input{structure.tex} % Include the file specifying the document structure and custom commands

%----------------------------------------------------------------------------------------
%	TITLE SECTION
%----------------------------------------------------------------------------------------
% Add a logo to the title page from the folder "Figures"





\title{	
	\includegraphics[width=0.25\textwidth]{C:/Users/saeedzou/Documents/GitHub/DeepLearning1401-01/Assignment 4/Theoretical/Figures/sharif_logo.jpg}\\
	\normalfont\normalsize
	\textsc{Sharif University of Technology}\\ % Your university, school and/or department name(s)
	\vspace{25pt} % Whitespace
	\rule{\linewidth}{0.5pt}\\ % Thin top horizontal rule
	\vspace{20pt} % Whitespace
	{\huge Deep Learning Assignment 3}\\ % The assignment title
	\vspace{12pt} % Whitespace
	\rule{\linewidth}{2pt}\\ % Thick bottom horizontal rule
	\vspace{12pt} % Whitespace
}

\author{\LARGE Saeedreza Zouashkiani} % Your name
\DeclareMathOperator{\E}{\mathbb{E}}
\DeclareMathOperator*{\argmin}{argmin}

\date{\normalsize\today} % Today's date (\today) or a custom date

\begin{document}
\maketitle % Print the title

\section{} %1
\subsection{} %1.1
We can see from the relations that if $z_t$ is zero, then the 
previous state $h_t$ is ignored and the current state is only
determined by new information. Meaning that it is the update gate.
On the other hand, if $r_t$ is zero, then the previous state's effect
is ignored and the new state is only determined by the previous state. 
Meaning that it is the reset gate.
\subsection{} %1.2
\begin{equation}
\begin{aligned}
	\frac{\partial J}{\partial W} =
	\sum_{k=1}^{t} \frac{\partial J^{(t)}}{\partial y_t} \frac{\partial y_t}{\partial W} =
	\sum_{k=1}^{t} \frac{\partial J^{(t)}}{\partial \hat{y}_k} \frac{\partial \hat{y}_k}{\partial h_t} \frac{\partial h_t}{\partial h_k} \frac{\partial h_k}{\partial W} 
\end{aligned}
\end{equation}
\begin{equation}
	\frac{\partial h_t}{\partial h_k} = \prod_{j=k+1}^{t} \frac{\partial h_j}{\partial h_{j-1}} 
\end{equation}
\begin{equation}
	\frac{\partial h_j}{\partial h_{j-1}}  = z_j + (1 - z_j) \frac{\partial \hat{h}_j}{\partial h_{j-1}} 
\end{equation}
We can see that if $z_j = 1$ then $\frac{\partial h_j}{\partial h_{j-1}} = 1 $, therefore vanishing gradient problem is solved. 

\end{document}
